\documentclass[12pt,a4paper]{article}

\input{../preamble_files/packages}
\input{../preamble_files/figures}
\input{../preamble_files/references}
\input{../preamble_files/shortcuts}

\pagestyle{fancy}
\lhead{Richard Whitehill}
\chead{MATH 513 -- HW \#}
\rhead{DATE}
\cfoot{\thepage~of~\pageref{LastPage}}

\newcommand{\prob}[2]{\textbf{#1)} #2}
\newenvironment{proof}
{
\textbf{\underline{Proof}} \\
}
{
\begin{flushright}
$\blacksquare$
\end{flushright}}

\setlength{\parskip}{\baselineskip}
\setlength{\parindent}{0pt}

\begin{document}

\prob{1}{For each of the following pairs of integers $a$ and $b$, determine their greatest common divisor, their least common multiple, and write their greatest command divisor in the form $ax + by$ for some integers $x$ and $y$}

(a) $a = 60$, $b = 17$

Observe that,
\begin{align*}
60 &= 3(17) + 9 \\
17 &= 1(9) + 8 \\
9 &= 1(8) + 1 \\
8 &= 8(1),
\end{align*}
so $\gcd(60,17) = 1$, and 
\begin{align*}
\gcd(60,17) &= 1 = 9 - (1*8) = 9 - (17 - 9) = 17 + 2(9) = -17 + 2(60 - 3*17) \\
&= 60(2) + 17(-7).
\end{align*}
Finally, we have $\lcm(60,17)\gcd(60,17) = \lcm(60,17) = 60(17) = 1020$. 

(b) $a = 11391$, $b = 5673$

Notice that
\begin{align*}
11391 &= 2(5673) + 45 \\
5673 &= 126(45) + 3 \\
45 &= 15(3),
\end{align*}
so $\gcd(11391,5673) = 3$, and
\begin{align*}
\gcd(11391,5673) &= 5673 - 126(45) = 5673 - 126(11391 - 2*5673) \\
&= 11391(-126) + 5673(253).
\end{align*}
Finally, we have $\lcm(11391,5673)\gcd(11391,5673) = 11391(5673)$ or $\lcm = 21540381$.

\prob{2}{Determine the value of $\varphi(n)$ for each integer $n \leq 15$, where $\varphi(n)$ denotes the Euler-$\varphi$ function.} \\

\begin{table}[h!]
\begin{center}
\begin{tabular}{|c|ccccccccccccccc|}
\hline
$n$ & 1 & 2 & 3 & 4 & 5 & 6 & 7 & 8 & 9 & 10 & 11 & 12 & 13 & 14 & 15 \\
\hline
$\varphi(n)$ & 1 & 1 & 2 & 2 & 2 & 2 & 6 & 4 & 6 & 4 & 10 & 4 & 12 & 6 & 8 \\
\hline
\end{tabular}
\end{center}
\end{table}

\prob{3}{Prove that if $p$ is prime, then $\sqrt{p}$ is not a rational number.} 

\begin{proof}
Suppose that $\sqrt{p}$ is a rational number. Then $\exists a,b \in \integers$ such that $\sqrt{p} = a/b$ and $\gcd(a,b) = 1$. That is, $a$ and $b$ are relatively prime and have no common factors. Thus, $a^2 = pb^2$, implying $p|a^2$. It follows then that $p|a$ or $a = px$ for some integer $x$. Hence, $a^2 = p^2x^2 = pb^2$ or $b^2 = px^2$, implying similarly that $p|b$. This is a contradiction, however, since we assumed that $\gcd(a,b) = 1$. We thus conclude that $\sqrt{p}$ is an irrational number.
\end{proof}

\prob{4}{Write down explicitly all the elements in the residue class of}

(a) $\integers/8\integers$

$\rightarrow$ $\integers/8\integers = \{0,1,2,3,4,5,6,7\}$

(b) $\integers/10\integers$

$\rightarrow$ $\integers/10\integers = \{0,1,2,3,4,5,6,7,8,9\}$

(c) $\integers/18\integers$

$\rightarrow$ $\integers/18\integers = \{0,1,2,3,4,5,6,7,8,9,10,11,12,13,14,15,16,17\}$

\prob{5}{Prove that there are infinitely many primes.}

\begin{proof}
Suppose that there are only a finite number of primes.
\end{proof}



\end{document}
