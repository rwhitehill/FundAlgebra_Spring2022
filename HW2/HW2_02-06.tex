\documentclass[12pt,a4paper]{article}

\input{../preamble_files/packages}
\input{../preamble_files/figures}
\input{../preamble_files/references}
\input{../preamble_files/shortcuts}

\pagestyle{fancy}
\lhead{Richard Whitehill}
\chead{MATH 513 -- HW 2}
\rhead{02/06/22}
\cfoot{\thepage~of~\pageref{LastPage}}

\newcommand{\prob}[2]{\textbf{#1)} #2}
\newenvironment{proof}
{
\textbf{\underline{Proof}} \\
}
{
\begin{flushright}
$\blacksquare$
\end{flushright}}

\setlength{\parskip}{\baselineskip}
\setlength{\parindent}{0pt}

\begin{document}

\prob{1}{For each of the following pairs of integers $a$ and $b$, determine their greatest common divisor, their least common multiple, and write their greatest command divisor in the form $ax + by$ for some integers $x$ and $y$}

(a) $a = 60$, $b = 17$

Observe that,
\begin{align*}
60 &= 3(17) + 9 \\
17 &= 1(9) + 8 \\
9 &= 1(8) + 1 \\
8 &= 8(1),
\end{align*}
so $\gcd(60,17) = 1$, and 
\begin{align*}
\gcd(60,17) &= 1 = 9 - (1*8) = 9 - (17 - 9) = 17 + 2(9) = -17 + 2(60 - 3*17) \\
&= 60(2) + 17(-7).
\end{align*}
Finally, we have $\lcm(60,17)\gcd(60,17) = \lcm(60,17) = 60(17) = 1020$. 

(b) $a = 11391$, $b = 5673$

Notice that
\begin{align*}
11391 &= 2(5673) + 45 \\
5673 &= 126(45) + 3 \\
45 &= 15(3),
\end{align*}
so $\gcd(11391,5673) = 3$, and
\begin{align*}
\gcd(11391,5673) &= 5673 - 126(45) = 5673 - 126(11391 - 2*5673) \\
&= 11391(-126) + 5673(253).
\end{align*}
Finally, we have $\lcm(11391,5673)\gcd(11391,5673) = 11391(5673)$ or $\lcm = 21540381$.

\prob{2}{Determine the value of $\varphi(n)$ for each integer $n \leq 15$, where $\varphi(n)$ denotes the Euler-$\varphi$ function.} \\

\begin{table}[h!]
\begin{center}
\begin{tabular}{|c|ccccccccccccccc|}
\hline
$n$ & 1 & 2 & 3 & 4 & 5 & 6 & 7 & 8 & 9 & 10 & 11 & 12 & 13 & 14 & 15 \\
\hline
$\varphi(n)$ & 1 & 1 & 2 & 2 & 2 & 2 & 6 & 4 & 6 & 4 & 10 & 4 & 12 & 6 & 8 \\
\hline
\end{tabular}
\end{center}
\end{table}

\prob{3}{Prove that if $p$ is prime, then $\sqrt{p}$ is not a rational number.} 

\begin{proof}
Suppose that $\sqrt{p}$ is a rational number. Then $\exists a,b \in \integers$ with $b \not= 0$ such that $\sqrt{p} = a/b$ and $\gcd(a,b) = 1$. That is, $a$ and $b$ are relatively prime and have no common factors. Thus, $a^2 = pb^2$, implying $p|a^2$. It follows then that $p|a$ or $a = px$ for some integer $x$. Hence, $a^2 = p^2x^2 = pb^2$ or $b^2 = px^2$, implying similarly that $p|b$. This is a contradiction, however, since we assumed that $\gcd(a,b) = 1$. We thus conclude that $\sqrt{p}$ is an irrational number.
\end{proof}

\prob{4}{Write down explicitly all the elements in the residue class of}

(a) $\integers/8\integers$

$\rightarrow$ $\integers/8\integers = \{\bar{0},\bar{1},\bar{2},\bar{3},\bar{4},\bar{5},\bar{6},\bar{7}\}$

(b) $\integers/10\integers$

$\rightarrow$ $\integers/10\integers = \{\bar{0},\bar{1},\bar{2},\bar{3},\bar{4},\bar{5},\bar{6},\bar{7},\bar{8},\bar{9}\}$

(c) $\integers/18\integers$

$\rightarrow$ $\integers/18\integers = \{\bar{0},\bar{1},\bar{2},\bar{3},\bar{4},\bar{5},\bar{6},\bar{7},\bar{8},\bar{9},\bar{10},\bar{11},\bar{12},\bar{13},\bar{14},\bar{15},\bar{16},\bar{17}\}$

\prob{5}{Prove that there are infinitely many primes.}

\begin{proof}
Suppose that there are only a finite number of primes $p_1,\hdots,p_k$. Consider the following integer,
\begin{align*}
M = p_1\hdots p_i \hdots p_k + 1.
\end{align*}
By the Fundamental Theorem of Arithmetic, $M$ is a composite number, which is written as a product of primes. That is, $M$ is divisible by at least one prime number. Notice, though, that $\gcd(M,p_i) = 1$ since $M = p_i(p_1 \hdots p_k) + 1$ for each $i = 1,\hdots,k$. Hence, we have a contradiction since $M$ is relatively prime to each prime number, implying that the set of prime numbers is in fact not finite.
\end{proof}

\prob{6}{Prove that if $\bar{a}$, $\bar{b} \in (\integers/n\integers)^{\times}$, then $\bar{a} \cdot \bar{b} \in (\integers/n\integers)^{\times}$.}

\begin{proof}
Since $\bar{a}$, $\bar{b} \in (\integers/n\integers)^{\times}$, there exist $\bar{c},\bar{d} \in (\integers/n\integers)^{\times}$ such that $\bar{a} \cdot \bar{c} = \bar{b} \cdot \bar{d} = \bar{1}$. Consider then $\bar{c} \cdot \bar{d}$. We have that $(\bar{a} \cdot \bar{b}) \cdot (\bar{c} \cdot \bar{d}) = (\bar{a} \cdot \bar{c}) \cdot (\bar{b} \cdot \bar{d}) = \bar{1} \cdot \bar{1} = \bar{1}$, implying that $\bar{a} \cdot \bar{b} \in (\integers/n\integers)^{\times}$.
\end{proof}

\prob{7[1.3.12]}{Let $n$, $m$, and $k$ all be positive integers. Assume that 
\begin{align*}
n | mk-1.
\end{align*}
Prove that $\gcd(n,m) = 1$.}

\begin{proof}
Notice that $ mk-1 = nl$ for some integer $l$. Hence, $1 = n(-l) + mk$. Since 1 is the least element in the positive integers and is an integer-linear combination of $n,m$, then $\gcd(n,m) = 1$.
\end{proof}

\prob{8[1.3.13]}{Let $a$, $b$, and $c$ be integers.}

(a) Prove that if $\gcd(a,b) = 1$ and $a|bc$, then $a|c$.

\begin{proof}
Notice that $1 = ax+by$ and that $bc = an$ for some integers $x$, $y$, and $n$. Hence $byc = (1 - ax)c = c - acx = an$ or $c = a(cx + n)$, implying that $a|c$.
\end{proof}

(b) Prove that if $\gcd(a,b) = 1$ and $\gcd(a,c) = 1$, then $\gcd(a,bc) = 1$.

\begin{proof}
Observe that $1 = an + bm$ and $1 = ak + cl$ for some $n,m,k,l \in \integers$. Thus $bc(ml) = (1-an)(1-ak) = 1 - a(n + k - ank)$ or $1 = ax + by$ where $x,y \in \integers$, proving that $\gcd(a,bc) = 1$.
\end{proof}

\prob{9[1.3.16]}{Let $a$ and $b$ be positive integers. Let $a = p_1^{\alpha_1}p_2^{\alpha_2} \hdots p_k^{\alpha_k}$ and $b = p_1^{\beta_1}p_2^{\beta_2}\hdots p_k^{\beta_k}$ where $\alpha_i,\beta_i \geq 0$ and, for $1 \leq i \leq k$, $p_i$ are distinct primes. Show that $\gcd(a,b) = p_1^{\gamma_1}p_2^{\gamma_2}\hdots p_k^{\gamma_k}$, where $\gamma_i = \min(\alpha_i,\beta_i)$. In particular, $a$ and $b$ are relatively prime if and only if they do not have any common prime divisors.}

\begin{proof}
Let $c = p_1^{\gamma_1}p_2^{\gamma_2}\hdots p_k^{\gamma_k}$. It is clear that $c|a$ and $c|b$ since $\gamma_i \leq \alpha_i,\beta_i$ for each $i$. We now prove that if $d|a$ and $d|b$ then $d|c$. Since $d|a$ and $d|b$, then it must be true that $d = p_1^{\delta_1}p_2^{\delta_2}\hdots p_k^{\delta_k}$, where $0 \leq \delta_i \leq \alpha_i,\beta_i$ and $\delta_i \in \integers$ for each $i$. Compare now $\delta_i$ and $\gamma_i = \min(\alpha_i,\beta_i)$. It is clear then that $\delta_i \leq \min(\alpha_i,\beta_i)$ for each $i$, so $d|c$ since $\gamma_i - \delta_i \geq 0$. Hence, $c = \gcd(a,b)$.

If $a$ and $b$ are relatively prime, then $\gcd(a,b) = 1$ and $\gamma_i = 0$ for each $i$, meaning that $\alpha_i$ or $\beta_i$ is zero. Hence $a$ and $b$ share no common prime divisors. Now suppose that $a$ and $b$ share no common divisors, then for each $i$ we have that at least one of $\alpha_i$ or $\beta_i$ is zero, meaning that $\gamma_i = 0$ and $\gcd(a,b) = 1$, implying that $a$ and $b$ are relatively prime.
\end{proof}

\prob{10[1.3.18]}{Let $a$ and $b$ be positive integers. What can you say about the product of $\gcd(a,b)$ and $\lcm(a,b)$? By looking at some examples, make a conjecture. Can you prove your conjecture?}

$\rightarrow$ It may be observed that $\gcd(a,b)\lcm(a,b) = ab$. Consider the following examples

\begin{table}[H]
\begin{center}
\begin{tabular}{c|c|c|c}
$(a,b)$ & $\gcd(a,b)$ & $\lcm(a,b)$ & $ab$ \\
\hline
(5,10) & 5 & 10 & 50 \\
(10,11) & 1 & 110 & 110 \\
(42,15) & 3 & 210 & 630 \\
\end{tabular}
\end{center}
\end{table}

We prove the conjecture below:

\begin{proof}
Notice that by definition if $c = \gcd(a,b)$ then $a = cn$ and $b = cm$ where $\gcd(n,m) = 1$. Obviously, if $d = \lcm(a,b)$ then $a|d$ and $b|d$. Hence, $d = cnm$ and $cd = c^2nm = ab$. That is, $\gcd(a,b)\lcm(a,b) = ab$.
\end{proof}


\end{document}
