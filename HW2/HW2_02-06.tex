\documentclass[12pt,a4paper]{article}

\input{../preamble_files/packages}
\input{../preamble_files/figures}
\input{../preamble_files/references}
\input{../preamble_files/shortcuts}

\pagestyle{fancy}
\lhead{Richard Whitehill}
\chead{MATH 513 -- HW \#}
\rhead{DATE}
\cfoot{\thepage~of~\pageref{LastPage}}

\newcommand{\prob}[2]{\textbf{#1)} #2}

\setlength{\parskip}{\baselineskip}
\setlength{\parindent}{0pt}

\begin{document}

\prob{1}{For each of the following pairs of integers $a$ and $b$, determine their greatest common divisor, their least common multiple, and write their greatest command divisor in the form $ax + by$ for some integers $x$ and $y$}

(a) $a = 60$, $b = 17$

Observe that,
\begin{align*}
60 &= 3(17) + 9 \\
17 &= 1(9) + 8 \\
9 &= 1(8) + 1 \\
8 &= 8(1),
\end{align*}
so $\gcd(60,17) = 1$, and 
\begin{align*}
\gcd(60,17) &= 1 = 9 - (1*8) = 9 - (17 - 9) = 17 + 2(9) = -17 + 2(60 - 3*17) \\
&= 60(2) + 17(-7).
\end{align*}
Finally, we have $\lcm(60,17)\gcd(60,17) = \lcm(60,17) = 60(17) = 1020$. 

(b) $a = 11391$, $b = 5673$

Notice that
\begin{align*}
11391 &= 2(5673) + 45 \\
5673 &= 126(45) + 3 \\
45 &= 15(3),
\end{align*}
so $\gcd(11391,5673) = 3$, and
\begin{align*}
\gcd(11391,5673) &= 5673 - 126(45) = 5673 - 126(11391 - 2*5673) \\
&= 11391(-126) + 5673(253).
\end{align*}
Finally, we have $\lcm(11391,5673)\gcd(11391,5673) = 11391(5673)$ or $\lcm = 21540381$.

\prob{2}{Determine the value of $\varphi(n)$ for each integer $n \leq 15$, where $\varphi(n)$ denotes the Euler-$\varphi$ function.} \\

\begin{table}
\begin{tabular}{|c|c|}
$n$ & $\varphi(n)$ \\
\hline
$n \leq 0$ & 0 \\
1 & 
2 &
3 &
4 &
5 &
6 &
7 &
8 &
9 &
10 &
11 &
12 &
13 &
14 &
15 &
\end{tabular}
\end{table}



\end{document}
