\documentclass[12pt,a4paper]{article}

\input{../preamble_files/packages}
\input{../preamble_files/figures}
\input{../preamble_files/references}
\input{../preamble_files/shortcuts}

\pagestyle{fancy}
\lhead{Richard Whitehill}
\chead{MATH 513 -- HW 3}
\rhead{02/13/22}
\cfoot{\thepage~of~\pageref{LastPage}}

\newcommand{\prob}[2]{\textbf{#1)} #2}
\newenvironment{proof}
{
\textbf{\underline{Proof}} \\
}
{
\begin{flushright}
$\blacksquare$
\end{flushright}}

\setlength{\parskip}{\baselineskip}
\setlength{\parindent}{0pt}

\begin{document}

\prob{1}{Find the orders of each element of the additive group $\integers/12\integers$.}

$\rightarrow$ For any group $(G,*)$ the order of $a \in G$ is defined as the least positive integer $n$ such that $a^n = \id$. For the additive group $\integers/12\integers$, we must find $n$ such that $\overline{na} = \bar{0}$. The following table summarizes the orders of each integer.

\begin{table}[H]
\begin{center}
\begin{tabular}{|c|c|c|c|c|c|c|c|c|c|c|c|c|}
\hline
element & $\bar{0}$ & $\bar{1}$ & $\bar{2}$ & $\bar{3}$ & $\bar{4}$ & $\bar{5}$ & $\bar{6}$ & $\bar{7}$ & $\bar{8}$ & $\bar{9}$ & $\overline{10}$ & $\overline{11}$ \\
\hline
order & 1 & 12 & 6 & 4 & 3 & 12 & 2 & 12 & 3 & 4 & 6 & 12 \\
\hline
\end{tabular}
\end{center}
\end{table}

\prob{2}{Find the orders of the following elements of the multiplicative group $(\integers/36\integers)^{\times}$: $\bar{1},~\overline{-1},~\bar{5},~\overline{-13},~\overline{17}$.}

$\rightarrow$ To find the orders of the elements of the group $(\integers/36\integers)^{\times}$, we determine the least $n$ such that $\bar{a}^n = \overline{a^n} = \bar{1}$. They are as follows:
\begin{align*}
\order{\bar{1}} &= 1 \\
\order{\overline{-1}} &= 2 \\
\order{\bar{5}} &= 6 \\
\order{\overline{-13}} &= 6 \\
\order{\overline{17}} &= 2
\end{align*}

\prob{3}{Let $(A,\star)$ and $(B,\diamond)$ be groups. Let $A \times B$ be the Cartesian product of $A$ and $B$. Define an operation $*$ on $A \times B$ by
\begin{align*}
(a_1,b_1) * (a_2,b_2) = (a_1 \star a_2, b_1 \diamond b_2).
\end{align*}
Show that $(A \times B,*)$ is a group.}

\begin{proof}
We prove the following properties:

i) Observe that 
\begin{align*}
[(a_1,b_1)*(a_2,b_2)]*(a_3,b_3) &= ((a_1 \star a_2)\star a_3, (b_1 \diamond b_2) \diamond b_3) \\
&= (a_1 \star (a_2\star a_3), b_1 \diamond (b_2 \diamond b_3)) \\
&=  (a_1,b_1)*[(a_2,b_2)*(a_3,b_3)]
\end{align*}
since $\star$ and $\diamond$ are associative binary operations on $A$ and $B$, respectively.

ii) Since $(A,\star)$ and $(B,\diamond)$ are groups, they must have identities $\id_{(A,\star)}$ and $\id_{(B,\diamond)}$. Thus, we see that $(A \times B,*)$ has an identity element $\id = (\id_{(A,\star)},\id_{(B,\diamond)})$ since $(a,b) * \id = (a \star \id_{(A,\star)},b \diamond \id_{(B,\diamond)}) = (a,b)$.

iii) Finally, since each element $a \in A$ and $b \in B$ have unique inverses $a^{-1}$ and $b^{-1}$ under $\star$ and $\diamond$, respectively, we see that $(a,b) \in A \times B$ has inverse $(a^{-1},b^{-1})$ under $*$, observing that $(a,b) * (a^{-1},b^{-1}) = (a \star a^{-1},b \diamond b^{-1}) = (\id_{(A,\star)},\id_{(B,\diamond)}) = \id$.
\end{proof}

\prob{4}{Let $A$ be the set of $2 \times 2$ matrices with real number entries. Let
\begin{align*}
M = \begin{pmatrix}
1 & 1 \\
0 & 1
\end{pmatrix}
\end{align*}
and let
\begin{align*}
B = \{X \in A : XM = MX\}
\end{align*}
}

1) Determine which of the following elements of $A$ lies in $B$
\begin{align*}
\begin{pmatrix}
1 & 1 \\
0 & 1
\end{pmatrix}
,
\begin{pmatrix}
1 & 1 \\
1 & 1
\end{pmatrix}
,
\begin{pmatrix}
1 & 1 \\
1 & 0
\end{pmatrix}
,
\begin{pmatrix}
0 & 0 \\
0 & 0
\end{pmatrix}
,
\begin{pmatrix}
1 & 0 \\
0 & 1
\end{pmatrix}
,
\begin{pmatrix}
0 & 1 \\
1 & 0
\end{pmatrix}
\end{align*}

$\rightarrow$ Observe the following:
\begin{align*}
\begin{pmatrix}
1 & 1 \\
0 & 1
\end{pmatrix}
M = M^2 = M\begin{pmatrix}
1 & 1 \\
0 & 1
\end{pmatrix} 
\end{align*}
\begin{align*}
\begin{pmatrix}
1 & 1 \\
1 & 1
\end{pmatrix}
\begin{pmatrix}
1 & 1 \\
0 & 1
\end{pmatrix} = \begin{pmatrix}
1 & 2 \\
1 & 2
\end{pmatrix},
\begin{pmatrix}
1 & 1 \\
0 & 1
\end{pmatrix}
\begin{pmatrix}
1 & 1 \\
1 & 1
\end{pmatrix} = \begin{pmatrix}
2 & 2 \\
1 & 1
\end{pmatrix}
\end{align*}
\begin{align*}
\begin{pmatrix}
1 & 1 \\
1 & 0
\end{pmatrix}
\begin{pmatrix}
1 & 1 \\
0 & 1
\end{pmatrix} = \begin{pmatrix}
1 & 2 \\
1 & 1
\end{pmatrix}
,
\begin{pmatrix}
1 & 1 \\
0 & 1
\end{pmatrix}
\begin{pmatrix}
1 & 1 \\
1 & 0
\end{pmatrix} = \begin{pmatrix}
2 & 1 \\
1 & 0
\end{pmatrix}
\end{align*}
\begin{align*}
\begin{pmatrix}
0 & 0 \\
0 & 0
\end{pmatrix}
M = M\begin{pmatrix}
0 & 0 \\
0 & 0
\end{pmatrix}
\end{align*}
\begin{align*}
\begin{pmatrix}
1 & 0 \\
0 & 1
\end{pmatrix}
M = \id_{2 \times 2} M = M \id_{2 \times 2} = M \begin{pmatrix}
1 & 0 \\
0 & 1
\end{pmatrix}
\end{align*}
\begin{align*}
\begin{pmatrix}
0 & 1 \\
1 & 0
\end{pmatrix}
\begin{pmatrix}
1 & 1 \\
0 & 1
\end{pmatrix} = \begin{pmatrix}
0 & 1 \\
1 & 1
\end{pmatrix}
,
\begin{pmatrix}
1 & 1 \\
0 & 1
\end{pmatrix}
\begin{pmatrix}
0 & 1 \\
1 & 0
\end{pmatrix} = \begin{pmatrix}
1 & 1 \\
1 & 0
\end{pmatrix}
\end{align*}
Hence, the matrices
\begin{align*}
\begin{pmatrix}
1 & 1 \\
0 & 1
\end{pmatrix}
,
\begin{pmatrix}
0 & 0 \\
0 & 0
\end{pmatrix}
,
\begin{pmatrix}
1 & 0 \\
0 & 1
\end{pmatrix}
\end{align*}
are elements of $B$.

2) Prove that if $P,Q \in B$, then $P+Q \in B$.

\begin{proof}
Notice that $PM = MP$ and $QM = MQ$. Hence, $PM + QM = MP + MQ$ or $(P+Q)M = M(P+Q)$. 
\end{proof}

3) Prove that if $P,Q \in B$, then $PQ \in B$.

\begin{proof}
Notice that $PM = MP$. Multiplying on the right by $Q$, we have $PMQ = MPQ$. Since $QM = MQ$, we have $PQM = MPQ$.
\end{proof}

4) Is the set $B$ with matrix addition a group?

\begin{proof}
We show that $B$ is a group under matrix addition. Since $B$ is a set of matrices and matrix addition is associative, it immediately follows that $B$ is associative under matrix addition. Additionally, we have trivially that
\begin{align*}
\id_{+} = \begin{pmatrix}
0 & 0 \\
0 & 0
\end{pmatrix}.
\end{align*}
Finally, it is easy to see that for any matrix $P$ that its inverse is $-P$ since $P + (-P) = P - P = 0$ (the zero matrix).
\end{proof}

5) Is the set $B$ with matrix multiplication a group?

\begin{proof}
It is easy to observe that $B$ is not a group under matrix multiplication. Matrix multiplication is associative generally and 
\begin{align*}
\id_{\times} = \begin{pmatrix}
1 & 0 \\
0 & 1
\end{pmatrix}
\end{align*}
is the identity for $2 \times 2$ matrix multiplication, but we cannot find a unique inverse for each $P \in B$ since $P$ is invertible iff $\det(P) \not= 0$. A counterexample may be 
\begin{align*}
P = \begin{pmatrix}
0 & 0 \\
0 & 0
\end{pmatrix}
\end{align*}
which has a determinant which is trivally zero. 
\end{proof}

Note: If we impose an additional restriction on $B$ such that only matrices which are invertible are elements of $B$, then $B$ would be a group under matrix multiplication.

\prob{5[2.1.2]}{Let $\theta$ be a real number and define 
\begin{align*}
R_{\theta} = \begin{bmatrix}
\cos{\theta} & -\sin{\theta} \\
\sin{\theta} & \cos{\theta}
\end{bmatrix}.
\end{align*}
}
(a) $R_{\theta}$ is called a rotation matrix. Can you explain why?

Consider rotating a point $(x,y) = (r\cos{\phi},r\sin{\phi})$ to a point $(x',y')$ by an angle $\theta$. Then
\begin{align*}
\begin{bmatrix}
x' \\ y'
\end{bmatrix}
&=
\begin{bmatrix}
r\cos{(\phi + \theta)} \\ r\sin{(\phi + \theta)}
\end{bmatrix}
= \begin{bmatrix}
r\cos{\phi}\cos{\theta} - r\sin{\phi}\sin{\theta} \\
r\cos{\phi}\sin{\theta} + r\sin{\phi}\cos{\theta}
\end{bmatrix}
=
\begin{bmatrix}
x\cos{\theta} - y\sin{\theta} \\
x\sin{\theta} + y\cos{\theta}
\end{bmatrix}
\\
&=
\underbrace{\begin{bmatrix}
\cos{\theta} & -\sin{\theta} \\
\sin{\theta} & \cos{\theta}
\end{bmatrix}}_{R_{\theta}}
\begin{bmatrix}
x \\ y
\end{bmatrix}
\end{align*}
It is seen then that the rotation matrix rotates points in the $xy$-plane through an angle $\theta$, or alternatively, it can be viewed as rotating the axes by $-\theta$.

(b) Show $R_{\theta}R_{\mu} = R_{?},~R_{\theta}^{-1} = R_{?}$.

Notice that 
\begin{align*}
R_{\theta}R_{\mu} &= \begin{bmatrix}
\cos{\theta} & -\sin{\theta} \\
\sin{\theta} & \cos{\theta}
\end{bmatrix}
\begin{bmatrix}
\cos{\mu} & -\sin{\mu} \\
\sin{\mu} & \cos{\mu}
\end{bmatrix} 
\\
&= 
\begin{bmatrix}
\cos{\theta}\cos{\mu} - \sin{\theta}\sin{\mu} & 
-[\cos{\theta}\sin{\mu} + \sin{\theta}\cos{\mu}] \\
\sin{\theta}\cos{\mu} + \cos{\theta}\sin{\mu} &
-\sin{\theta}\sin{\mu} + \cos{\theta}\cos{\mu}
\end{bmatrix}
\\
&= \begin{bmatrix}
\cos{(\theta + \mu)} & -\sin{(\theta + \mu)} \\
\sin{(\theta + \mu)} & \cos{(\theta + \mu)}
\end{bmatrix}
= R_{\theta + \mu},
\end{align*}
and
\begin{align*}
R_{\theta}R_{-\theta} = \begin{bmatrix}
\cos{(\theta - \theta)} & -\sin{(\theta - \theta)} \\
\sin{(\theta - \theta)} & \cos{(\theta - \theta)}
\end{bmatrix}
=
\begin{bmatrix}
1 & 0 \\
0 & 1
\end{bmatrix}
= \id,
\end{align*}
meaning $R_{\theta}R_{\mu} = R_{\theta + \mu}$ and $R_{\theta}^{-1} = R_{-\theta}$.

(c) Let $G = \{R_{\theta}|\theta\in\reals\}$. Show that $G$ is a group under matrix multiplication.

\begin{proof}
Since the elements of $G$ are matrices and matrix multiplication is associative, it follows that $G$ is associative. We also see that $R_{0}$ is the identity element in $G$ since it is also the $2 \times 2$ identity matrix. We showed in part (b) that $R_{\theta}^{-1} = R_{-\theta}$. It also suffices to show that $\det(R_{\theta}) = \cos^2{\theta} + \sin^2{\theta} = 1 \not= 0$, implying that each matrix in $G$ has an inverse. 
\end{proof}

\prob{6[2.1.3]}{Let $\integers$ denote the set of integers, and let 
\begin{align*}
G = \{\begin{bmatrix}
1 & a & 0 \\
0 & 1 & 0 \\
0 & 0 & 0
\end{bmatrix} \Big| a \in \integers\}.
\end{align*}
Prove that $G$ together with the usual matrix multiplication forms a group.}

\begin{proof}
Observe that $G$ is associative since its elements are matrices and matrix multiplication is generally associative. Also, the element of $G$ with $a = 1$ serves as the identity since
\begin{align*}
\begin{bmatrix}
1 & 0 & 0 \\
0 & 1 & 0 \\
0 & 0 & 0
\end{bmatrix}
\begin{bmatrix}
1 & b & 0 \\
0 & 1 & 0 \\
0 & 0 & 0
\end{bmatrix}
=
\begin{bmatrix}
1 & b & 0 \\
0 & 1 & 0 \\
0 & 0 & 0
\end{bmatrix}
\end{align*}
for any $b \in \integers$. Finally, we see that 
\begin{align*}
\begin{vmatrix}
1 & a & 0 \\
0 & 1 & 0 \\
0 & 0 & 0
\end{vmatrix}
= 1,
\end{align*}
so every matrix in $G$ has a unique inverse.
\end{proof}

\prob{7[2.2.1]}{Let $G$ be a group. Prove that $(ab)^{-1} = a^{-1}b^{-1}$ for all $a$ and $b$ in $G$ if and only if $G$ is abelian.}

\begin{proof}
($\Leftarrow$) We have proven that $(ab)^{-1} = b^{-1}a^{-1}$ for any group. If $G$ is abelian, then it follows that $b^{-1}a^{-1} = a^{-1}b^{-1}$.

($\Rightarrow$) Suppose that $(ab)^{-1} = a^{-1}b^{-1}$. We have also proven that $(ab)^{-1} = b^{-1}a^{-1}$. Hence, $(ab)^{-1} = (ba)^{-1}$ for all $a,b \in G$. This implies that $ab = ba$ (since each element has a unique inverse) and that $G$ is abelian.
\end{proof}

\prob{8[2.2.2]}{Let $G$ be a group. Show that, for all $a,b\in G$, we have $(ab)^2 = a^2b^2$ if and only if $G$ is abelian.}

\begin{proof}
($\Leftarrow$) Let $G$ be an abelian group. Then $ab = ba$ for all $a,b \in G$, and $(ab)^2 = abab = aabb = a^2b^2$.

($\Rightarrow$) Assume for all $a,b \in G$ that $(ab)^2 = a^2b^2$. Then $abab = aabb$. Multiplying on the left by $a^{-1}$ and on the right by $b^{-1}$ we have $a^{-1}(abab)b^{-1} = ba$ and $a^{-1}(aabb)b^{-1} = ab$. That is, $ba = ab$ or $G$ is abelian.
\end{proof}

\prob{9[2.2.3]}{If $G$ is a group in which $a^2 = \id$ for all $a \in G$, show that $G$ is abelian.}

\begin{proof}
Observe that each element in $G$ is its own inverse. That is $a=a^{-1}$. Thus, $ab = (ab)^{-1} = b^{-1}a^{-1} = ba$, proving that $G$ is abelian.
\end{proof}

\prob{10[2.2.4]}{}

(a) If $G$ is a finite group of even order, show that there must be an element $a \not= \id$, such that $a^{-1} = a$. 

\begin{proof}
Since $G$ is finite and even we have $G = \{\id,a_2,a_3,\hdots,a_n\}$ where $n$ is even. Suppose that for each element $a \not= \id$ that $a^{-1} \not= a$. However, there are $n-1$ elements which are not the identity, and since $2 \not|~n-1$, it is impossible that each element has a unique inverse. Thus, it must be the case that there exists at least one $a \not= \id$ in $G$ such that $a^{-1} = a$.
\end{proof}

(b) Give an example to show that the conclusion of part (a) does not hold for groups of odd order.

$\rightarrow$ Consider a cyclic group of order 3: $\{\id,a,a^2\}$. It is clear that $a^{-1} = a^2$ and $(a^{2})^{-1} = a$. As a concrete example consider $(\integers/5\integers)^{\times}$ and $\left<\overline{2}\right> = \{\overline{2},\overline{4},\overline{1}\}$. This is a group of order three with $\overline{1}^{-1} = \overline{1}$, $\overline{2}^{-1} = \overline{4}$, and $\overline{4}^{-1} = \overline{2}$.

\end{document}
