\documentclass[12pt,a4paper]{article}

\input{../preamble_files/packages}
\input{../preamble_files/figures}
\input{../preamble_files/references}
\input{../preamble_files/shortcuts}

\pagestyle{fancy}
\lhead{Richard Whitehill}
\chead{MATH 513 -- HW 4}
\rhead{02/20/22}
\cfoot{\thepage~of~\pageref{LastPage}}

\newcommand{\prob}[2]{\textbf{#1)} #2}
\newenvironment{proof}
{
\textbf{\underline{Proof}} \\
}
{
\begin{flushright}
$\blacksquare$
\end{flushright}}

\setlength{\parskip}{\baselineskip}
\setlength{\parindent}{0pt}

\begin{document}

\prob{1}{Compute the order of each element in the dihedral group $D_6$ and $D_8$.}

$\rightarrow$ Consider the symmetries of a triangle given in the figure below and let $D_6 = \{R_{0},R_{60},\\ R_{120},D_1,D_2,D_3\}$. 
\bef
\includegraphics[scale=0.3]{fig1.png}
\eef

Then we have the following
\begin{table}[H]
 \begin{center}
  \begin{tabular}{|c|c|c|c|c|c|c|}
   \hline
   element & $R_{0}$ & $R_{60}$ & $R_{120}$ & $D_1$ & $D_2$ & $D_3$ \\
   \hline
   order & 1 & 3 & 3 & 2 & 2 & 2 \\
   \hline
  \end{tabular}
 \end{center}
\end{table}
For each symmetry operation, we consider the minimum number of applications that are needed to return to the original configuration. Obviously, $R_{0}$ is the identity, which always has order of 1 for any group, and the reflections across diagonals through the vertices and center must be performed twice.

Now consider the symmetries of a square shown below and let $D_8 = \{R_{0},R_{90},R_{180},R_{270},H,\\V,D,D'\}$. 
\bef
\includegraphics[scale=0.3]{fig2.png}
\eef

Then it follows that
\begin{table}[H]
 \begin{center}
  \begin{tabular}{|c|c|c|c|c|c|c|c|c|}
   \hline
   element & $R_{0}$ & $R_{90}$ & $R_{180}$ & $R_{270}$ & $H$ & $V$ & $D$ & $D$ \\
   \hline
   order & 1 & 4 & 2 & 4 & 2 & 2 & 2 & 2 \\
   \hline
  \end{tabular}
 \end{center}
\end{table}

\prob{2}{Show that $\left<a,b | a^2=b^2=(ab)^2=\id \right>$ give a presentation for $D_{2n}$ in terms of the two generators $a = s$ and $b = sr$ of order $2$.}

\prob{3[1.1.1]}{Complete the multiplication table for $D_8$. Find at least one interesting pattern in the table.}

\begin{table}[H]
 \begin{center}
  \begin{tabular}{c|cccccccc}
    & $R_{0}$ & $R_{90}$ & $R_{180}$ & $R_{270}$ & $H$ & $V$ & $D$ & $D'$ \\
  \hline
  $R_{0}$ & $R_{0}$ & $R_{90}$ & $R_{180}$ & $R_{270}$ & $H$ & $V$ & $D$ & $D'$ \\ 
  $R_{90}$ & $R_{90}$ & $R_{180}$ & $R_{270}$ & $R_{0}$ & $D'$ & $D$ & $H$ & $V$ \\
  $R_{180}$ & $R_{180}$ & $R_{270}$ & $R_{0}$ & $R_{90}$ & $V$ & $H$ & $D'$ & $D$ \\
  $R_{270}$ & $R_{270}$ & $R_{0}$ & $R_{90}$ & $R_{180}$ & $D$ & $D'$ & $V$ & $H$ \\
  $H$ & $H$ & $D$ & $V$ & $D'$ & $R_{0}$ & $R_{180}$ & $R_{90}$ & $R_{270}$ \\
  $V$ & $V$ & $D'$ & $H$ & $D$ & $R_{180}$ & $R_{0}$ & $R_{270}$ & $R_{90}$ \\
  $D$ & $D$ & $V$ & $D'$ & $H$ & $R_{270}$ & $R_{90}$ & $R_{0}$ & $R_{180}$ \\
  $D'$ & $D'$ & $H$ & $D$ & $V$ & $R_{90}$ & $R_{270}$ & $R_{180}$ & $R_{0}$ \\
  \end{tabular}
 \end{center}
\end{table}

*Something interesting*

\prob{4[1.1.4]}{}

(a) List the symmetries of a rectangle.

The symmetries are given in the second figure in problem 1 can be put into the set $D_8$, which is also given in problem 1.

(b) Write the multiplication table for the symmetries of a rectangle.

The multiplication table for the symmetries of a rectangle are shown in the multiplication table in problem 3.

\prob{5[1.1.5]}{}

(a) Find the center of $D_8$

The center of a group $G$ is the set of all elements $a \in G$ such that $ab = ba$ for all $b \in G$ (i.e. $a$ commutes with all elements of $G$). We can read from the table in problem 3 what elements are in the center of $D_8$:
\begin{align*}
\mathbf{Z}(D_8) = \{R_0,R_{180}\}
\end{align*}

(b) Find $\mathbf{C}_{D_8}(R_{90})$ and $\mathbf{C}_{D_8}(H)$.

The centralizer of an element $a \in G$ is the set of all elements $b \in G$ such that $a$ and $b$ commute:
\begin{align*}
\mathbf{C}_{D_8}(R_{90}) = \{R_{0},R_{90},R_{180},R_{270}\}
\end{align*}
\begin{align*}
\mathbf{C}_{D_8}(H) = \{H,R_{180},R_{0},V\}
\end{align*}

\prob{6[1.1.6]}{Let $D_6$ denote the set of symmetries of an equilateral triangle. Find the multiplication table for $D_6$. What is the center of $D_6$?}

\begin{table}
 \begin{center}
  \begin{tabular}{c|cccccc}
   & $R_{0}$ & $R_{60}$ & $R_{120}$ & $D_1$ & $D_2$ & $D_3$ \\
  \hline
  $R_{0}$ & $R_{0}$ & $R_{60}$ & $R_{120}$ & $D_1$ & $D_2$ & $D_3$ \\
  $R_{60}$ & $R_{60}$ & $R_{120}$ & $R_{0}$ & $D_2$ & $D_3$ & $D_1$ \\
  $R_{120}$ & $R_{120}$ & $R_{0}$ & $R_{60}$ & $D_3$ & $D_1$ & $D_2$ \\
  $D_1$ & $D_1$ & $D_3$ & $D_2$ & $R_{0}$ & $R_{120}$ & $R_{60}$ \\
  $D_2$ & $D_2$ & $D_1$ & $D_3$ & $R_{60}$ & $R_{0}$ & $R_{120}$ \\
  $D_3$ & $D_3$ & $D_2$ & $D_1$ & $R_{120}$ & $R_{60}$ & $R_{0}$ \\
  \end{tabular}
 \end{center}
\end{table}

\prob{7[1.2.4]}{Let $\sigma = (1~3~5)(2~4)$ and $\tau = (1~5)(2~3)$ be elements of $S_5$. Find $\sigma^2$, $\sigma\tau$, $\tau\sigma$, and $\tau\sigma^2$.}

\begin{align*}
\sigma^2 &= [(1~3~5)(2~4)][(1~3~5)(2~4)] = (1~5~3) \\
\sigma\tau &= [(1~3~5)(2~4)][(1~5)(2~3)] = (5~3~4~2) \\
\tau\sigma &= [(1~5)(2~3)][(1~3~5)(2~4)] = (1~2~4~3) \\
\tau\sigma^2 &= [(1~5)(2~3)](1~5~3) = (5~2~3)
\end{align*}

\prob{8[1.2.5]}{Construct a complete multiplication table for $S_3$. What is the center of $S_3$? If $f = (1~2~3)$, what is $\mathbf{C}_{S_3}(f)$, the centralizer of $f$ in $S_3$?}

$\rightarrow$ We know that $S_3 = {\rm Perm}(\{1,2,3\}) = \{\id,(2~3),(1~2),(1~2~3),(1~3~2),(1~3)\}$. Hence,
\begin{table}[H]
 \begin{center}
  \begin{tabular}{c|cccccc}
   & $\id$ & (2~3) & (1~2) & (1~2~3) & (1~3~2) & (1~3) \\
  \hline
  $\id$ & $\id$ & (2 3) & (1 2) & (1 2 3) & (1 3 2) & (1 3) \\
  (2 3) & (2 3) & $\id$ & (1 3 2) & (1 3) & (1 2) & (1 2 3) \\
  (1 2) & (1 2) & (1 2 3) & $\id$ & (2 3) & (1 3) & (1 3 2) \\
  (1 2 3) & (1 2 3) & (1 2) & (1 3) & (1 3 2) & $\id$ & (2 3) \\
  (1 3 2) & (1 3 2) & (1 3) & (2 3) & $\id$ & (1 2 3) & (1 2) \\
  (1 3) & (1 3) & (1 3 2) & (1 2 3) & (1 2) & (2 3) & $\id$ \\
  \end{tabular}
 \end{center}
\end{table}

and 
\begin{align*}
\mathbf{C}_{S_3}(f) = \{\id,(1~2~3),(1~3~2)\}
\end{align*}

\prob{9[1.2.6]}{Let $f = (1~2~3) \in S_3$. Find the maps in the following sequence
\begin{align*}
\id_{[3]},f,f^2,f^3,f^4,f^5,\hdots
\end{align*}
Do you see a pattern?}

Notice that from the table we see that
\begin{align*}
f^{0} &= \id_{[3]} \\
f^{1} &= (1~2~3) \\
f^{2} &= (1~2~3)(1~2~3) = (1~3~2) \\
f^{3} &= (1~2~3)(1~3~2) = \id_{[3]} \\
\vdots
\end{align*}
It is seen that this sequence essentially lists the elements of the cyclic group $<f> = \{f,f^2,f^3=\id_{[3]}\}$.


\end{document}
