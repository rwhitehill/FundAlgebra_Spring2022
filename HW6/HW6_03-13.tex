\documentclass[12pt,a4paper]{article}

\input{../preamble_files/packages}
\input{../preamble_files/figures}
\input{../preamble_files/references}
\input{../preamble_files/shortcuts}

\pagestyle{fancy}
\lhead{Richard Whitehill}
\chead{MATH 513 -- HW 6}
\rhead{03/13/22}
\cfoot{\thepage~of~\pageref{LastPage}}

\newcommand{\prob}[2]{\textbf{#1)} #2}
\newenvironment{proof}
{
\textbf{\underline{Proof}} \\
}
{
\begin{flushright}
$\blacksquare$
\end{flushright}}

\setlength{\parskip}{\baselineskip}
\setlength{\parindent}{0pt}

\begin{document}

\prob{1.7.2}{Show that the additive group $\integers$ acts on itself by $z \cdot a = z + a$ for all $z,a \in \integers$.}

$\rightarrow$ Define the map $\integers \times \integers \rightarrow \integers$ such that $z \cdot a \mapsto z + a$. We prove that this map is a group action on $\integers$.

\begin{proof}
(1) Let $x,y,a \in \integers$. Then $x \cdot (y \cdot a) = x \cdot (y + a) = x + (y + a)$ and $(x + y) \cdot a = (x + y) + a$. Obviously these are equal since the integers are associative under addition.

(2) Recall that $0 = \id_{\integers}$. Thus, $0 \cdot a = 0 + a = a$ for any integer $a$.
\end{proof}

\prob{1.7.6}{Prove that a group $G$ acts faithfully on a set $A$ iff the kernel of the action is the set consisting only of the identity.}

\begin{proof}
($\Rightarrow$) Suppose that $G$ acts faithfully on $A$. That is, if $g_1,g_2$ are distinct elements of $G$, then $\sigma_{g_1} \not= \sigma_{g_2}$. By the definition of a group action, $\id$ is in the kernel of the group action. However, it is the only element in the kernel since if $g \not= \id$, then $\sigma_{g} \not= \sigma_{\id}$. Hence, there exists at least one element $b \in A$ such that $\sigma_g(b) \not= b$.

($\Leftarrow$) Suppose that the kernel of a group action $\{g \in G | \forall a \in A, ga = a\} = \{\id\}$ but that the group action is not faithful. Then there exist two distinct group elements $g_1,g_2$ such that $\sigma_{g_1} = \sigma_{g_2}$. Then $\sigma_{g_2^{-1}} \circ \sigma_{g_1} = \id$, which would imply that $g_2^{-1}g_1$ is in the kernel of the group action, but this is a contradiction since $g_2^{-1}g_1 \not= \id$. Hence, the group action must be faithful.
\end{proof}

\prob{1.7.14}{Let $G$ be a group and let $A = G$. Show that if $G$ is non-abelian then the maps defined by $g \cdot a = ag$ for all $g,a \in G$ do not satisfy the axioms of a (left) group action of $G$ on itself.}

\begin{proof}
It is clear that the second requirement of the map is satisfied since $\id \cdot a = a\id = a$. However, the first is not since if $g_1 \cdot (g_2 \cdot a) = (g_1g_2) \cdot a$ then $ag_2g_1 = a(g_1g_2)$ or $g_1g_2 = g_2g_1$ would contradict our assumption that the group $G$ is non-abelian. Hence, this cannot be a (left) group action.
\end{proof}

\prob{1.7.15}{Let $G$ be any group and let $A = G$. Show that the maps defined by $g \cdot a = ag^{-1}$ for all $g,a \in G$ do satisfy the axioms of a (left) group action of $G$ on itself.}

\begin{proof}
(1) Notice that $g_1 \cdot (g_2 \cdot a) = g_1 \cdot (ag_2^{-1}) = ag_2^{-1}g_1^{-1} = a(g_1g_2)^{-1} = (g_1g_2) \cdot a$.

(2) Also observe that $\id \cdot a = a\id^{-1} = a\id = a$.
\end{proof}

\prob{1.7.16}{Let $G$ be any group and let $A = G$. Show that the maps defined by $g \cdot a = gag^{-1}$ for all $g,a \in G$ do satisfy the axioms of a (left) group action (this action of $G$ on itself is called \textit{conjugation})}

\begin{proof}
(1) We see that $g_1 \cdot (g_2 \cdot a) = g_1 \cdot g_2ag_2^{-1} = g_1g_2ag_2^{-1}g_2^{-1}$ and $(g_1g_2) \cdot a = g_1g_2a(g_1g_2)^{1}$, implying that this map satisfies the first property of a group action.

(2) It is trivial to see that $\id \cdot a = a$.
\end{proof}

\prob{2.2.4}{For each of $S_3$, $D_8$, and $Q_8$ compute the centralizers of each element and find the center of each group.}

$\rightarrow$ Recall $S_3 = \{\id_{[3]},(2~3),(1~2),(1~3),(1~2~3),(1~3~2)\}$.
Thus, from previous homeworks where the multiplication tables were computed, we have the centralizers of the following elements:
\begin{align*}
\id_{[3]}&: S_3 \\
(2~3)&: \{\id_{[3]},(2~3)\} \\
(1~2)&: \{\id_{[3]},(1~2)\} \\
(1~3)&: \{\id_{[3]},(1~3)\} \\
(1~2~3)&: \{\id_{[3]},(1~2~3),(1~3~2)\} \\
(1~3~2)&: \{\id_{[3]},(1~2~3,(1~3~2))\}
\end{align*}
and the center of $S_3$
\begin{align*}
Z(S_3) = \{\id_{[3]}\}
\end{align*}

Next we have the dihedral group of order 8: $D_8 = \{R_{0},R_{90},R_{180},R_{270},H,V,D,D'\}$.
Again, we have computed the multiplication tables for $D_8$ previously, giving the centralizers for the following elements as:
\begin{align*}
R_{0}&: D_8 \\
R_{90}&: \{R_{0},R_{90},R_{180},R_{270}\} \\ 
R_{180}&: D_8 \\
R_{270}&: \{R_{0},R_{90},R_{180},R_{270}\} \\
H&: \{R_{0},H,R_{180}\} \\
V&: \{R_{0},V,R_{180}\} \\
D&: \{R_{0},D,R_{180}\} \\
D'&: \{R_{0},D',R_{180}\}
\end{align*}
and the center
\begin{align*}
Z(D_8) = \{\id,R_{180}\}
\end{align*}
which is proven below.

Finally, we have the quaternion group $Q_8 = \{\pm 1, \pm i, \pm j, \pm k\}$. The centralizers are straightforward to glean from the definition of the group:
\begin{align*}
\pm 1&: Q_8 \\
\pm i&: \{\pm 1, \pm i\} \\
\pm j&: \{\pm 1, \pm j\} \\
\pm k&: \{\pm 1, \pm k\}
\end{align*}
and the center of the group is 
\begin{align*}
Z(Q_8) = \{\pm 1\}
\end{align*}

\prob{2.2.5}{In each of parts (a) to (c) show that for the specified group $G$ and subgroup $A$ of $G$, $C_{G}(A) = A$ and $N_G(A) = G$.}

(a) $G = S_3$ and $A = \{1,(123),(132)\}$.

Recall $S_3 = \{\id_{[3]},(2~3),(1~2),(1~3),(1~2~3),(1~3~2)\}$. Then, 
\begin{align*}
C_{G}(A) &= \{\sigma \in S_3 | \forall \tau \in A, \sigma\tau = \tau\sigma \} = C_G(\id_{[3]}) \cap C_G((123)) \cap C_G((132)) = \{\id_{[3]},(123),(132)\} = A \\
N_{G}(A) &= \{\sigma \in S_3 | \sigma A \sigma^{-1} = A\} = \{\id_{[3]},(12),(13)\} \cup A = G
\end{align*}
Note that $C_{G}(A) \leq N_G(A)$, so by definition $C_G(A) \subset N_G(A)$.

(c) $G = D_{10}$ and $A = \{1,r,r^2,r^3,r^4\}$.

Using the generators and relations description of the dihedral group we have $D_{10} = \langle r,s | r^5 = s^2 = \id, rs = sr^{4} \rangle$, meaning
\begin{align*}
C_G(A) &= \{x \in D_{10} | \forall a \in A ax = xa \} = \{1,r^2,r^3,r^4,r^5\} = A  \\
N_G(A) &= \{x \in D_{10} | xAx^{-1} = A \} = \{s,sr,sr^2,sr^3,sr^4\} \cup A = G
\end{align*}
Note that if $a \in A$ then $sr^{k}a(sr^{k})^{-1} = sr^{k}ar^{-k}s = sr^{k}r^{-k}as = sas = a^{-1}$. Observe that $A = \langle r \rangle$, so if $a \in A$, then $a^{-1} \in A$, which gives us the first part of the union for $N_G(A)$.

\prob{2.2.6}{Let $H$ be a subgroup of the group $G$.}

(a) Show that $H \leq N_G(H)$. Give an example to show that this is not necessarily true if $H$ is not a subgroup.

\begin{proof}
Recall that $N_{G}(H) = \{g \in G | gHg^{-1} = H\}$ and $gHg^{-1} = \{ghg^{-1} | h \in H\}$. We know that $H,N_G(H)$ are both subgroups of $G$ so we must simply show that $H \subset N_G(H)$. That is, we must prove that $hHh^{-1} = H$. This is simple to observe since $H$ is closed under products and if $a \in H$, then $a \in hHh^{-1}$ since $h(h^{-1}ah)h^{-1} = a$.
\end{proof}

$\rightarrow$ The above result hinges on the assumption that $H$ is a subgroup of $G$. Consider the following example: $G = \integers/3\integers$ and $H = \{\bar{1}\}$. Obviously, $H$ is not a subgroup of $G$ since it is not closed under addition and $N_G(H) = \integers/3\integers$.

(b) Show that $H \leq C_G(H)$ iff H is abelian.

\begin{proof}
($\Rightarrow$) Suppose that $H \leq C_G(H)$. Then it is clear that for any two elements $h,a \in H$ that $ha = ah$, implying that $H$ is abelian. 

($\Leftarrow$) Now, suppose that $H$ is abelian. Then, for any two $h,a \in H$ we have $ha = ah$. Thus, fixing $h$, it is clear that $h \in C_G(H)$, and since we have shown previously that $C_G(H) \leq G$, it immediately follows that $H \leq C_G(H)$.
\end{proof}

\prob{2.2.7}{Let $n \in \integers$ with $n \geq 3$. Prove the following:}

(a) $Z(D_{2n}) = \{1\}$ if $n$ is odd.

$\rightarrow$ Recall the generators and relations representation of the dihedral group: $D_{2n} = \langle r,s | r^n = s^2 = \id, rs = sr^{-1} \rangle$. Obviously, $\id$ commutes with any element, implying that $\id \in Z(D_{2n})$. Next, we see $s$ does not commute with any element of the dihedral group since $rs = sr^{-1}$. Finally, we see that $r^{k}sr^{m} = sr^{-k}r^{m} = sr^{m}r^{n-k}$. We must have $k = n-k$ for these elements to commute. That is, $n = 2k$. However, if $n$ is odd, then $n$ cannot be written as twice an integer. Thus, we arrive at the conclusion that if $n$ is odd, then $Z(D_{2n}) = \{\id\}$.

(b) $Z(D_{2n}) = \{1,r^k\}$ if $n = 2k$.

From the reasoning above, we see that if $n$ is even, then $n = 2k$ for some integer $k$, implying that $r^k \in Z(D_{2n})$ and  $Z(D_{2n}) = \{\id,r^k\}$


\end{document}
