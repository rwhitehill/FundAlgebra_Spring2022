\documentclass[12pt,a4paper]{article}

\input{../preamble_files/packages}
\input{../preamble_files/figures}
\input{../preamble_files/references}
\input{../preamble_files/shortcuts}

\pagestyle{fancy}
\lhead{Richard Whitehill}
\chead{MATH 513 -- HW 10}
\rhead{04/24/22}
\cfoot{\thepage~of~\pageref{LastPage}}

\newcommand{\prob}[2]{\textbf{#1)} #2}
\newenvironment{proof}
{
\textbf{\underline{Proof}} \\
}
{
\hfill $\blacksquare$
}

\setlength{\parskip}{\baselineskip}
\setlength{\parindent}{0pt}

\begin{document}

\prob{7.4.1}{
    Let $L_{j}$ be the left ideal of $M_{n}(R)$ consisting of arbitrary entries in the $j^{\rm th}$ column and zero in all other entries and let $E_{ij}$ be the element of $M_{n}(R)$ whose $i,\,j$ entry is 1 and whose other entries are all 0.
Prove that $L_{j} = M_{n}(R)E_{ij}$ for any $i$.
}

\begin{proof}
    Let 
    \begin{align*}
        \delta_{ij} = \begin{cases}
            1 & i = j \\
            0 & i \ne j
        \end{cases}
    .\end{align*}
    be the kronecker delta symbol.
    This is just notation that will simplify the matrix multiplication visually.
    Thus, if $M \in M_{n}(R)$, then 
    \begin{align*}
        (ME_{ij})_{lm} = \sum_{k=1}^{n} M_{lk}(E_{ij})_{km} &= \sum_{k=1}^{n} M_{lk}\delta_{ik}\delta_{jm} \\
        &= M_{li}\delta_{jm}.
    .\end{align*}
    This tells us that the entries are only nonzero if $m = j$, meaning that only the $j^{\rm th}$ column on $ME_{ij}$ is nonzero, with entries $M_{li}$, where $l$ is the row label.
    Hence, $ME_{ij} \in L_{j}$ for all $M \in M$.

    Now, we must prove that $L_{j} \subset M_{n}(R)E_{ij}$.
    This is easy to do by construction: we translate from $L_{j}$ to $M_{n}(R)$ by taking the $j^{\rm th}$ column of the element of $L_{j}$ and making it the $j^{\rm th}$ row of some element in $M_{n}(R)$.
    Since we know then that multiplying this element in $M_{n}(R)$ by $E_{ij}$ gives $L_{j}$, it is clear that $L_{j} \subset M_{n}(R)E_{ij}$.
\end{proof}

\prob{7.4.4}{
Assume $R$ is commutative.
Prove that $R$ is a field iff 0 is a maximal ideal.
}

\begin{proof}
    ($\implies$) Suppose that $0$ is a maximal ideal of $R$.
    Then $R/\{ 0 \}$ is a field.
    We can construct a trivial isomorphism between this quotient ring and $R$ as follows:
    \begin{align*}
        \varphi&: R / \{ 0 \} \to R \\
        \varphi&: \overline{r} \mapsto r
    .\end{align*}
    Since $R \cong R / \{ 0 \} $, we know that $R$ is a field too.

    ($\implies$) We know that a field only has $\{ 0 \} $ and itself as maximal ideals, so this direction of the proof is already done.
\end{proof}

\prob{7.4.7}{
    Let $R$ be a commutative ring with 1.
Prove that the principal ideal generated by $x$ in the polynomial ring $R[x]$ is a prime ideal iff $R$ is an integral domain.
Prove that $(x)$ is a maximal ideal iff $R$ is a field.
}

\begin{proof}
    Consider the ring homomorphism $\varphi: R[x] \to R$ defined by $\varphi: p(x) \mapsto p(0)$.
    Then, the $\ker \varphi = (x)$ since $p(0)$ if and only if $a_0 = 0$, where $a_0$ is the constant term of $p(x)$.
    If the constant term is zero, then by the division algorithm, we can write $p(x) = xq(x) \in (x)$.
    By the first isomorphism theorem for rings, we know that $R[x]/(x) \cong R$.
    
    It is known that $R/P$ is an integral domain if and only if $P$ is a prime ideal in $R$.
    Hence, the proposition in the problem statement follows immediately from the above isomorphism.

    It is also known that $R/M$ is a field if and only if $M$ is an maximal ideal, so in the same manner as above, we know immediately that the second proposition follows.
\end{proof}

\prob{7.4.8}{
Let $R$ be an integral domain.
Prove that $(a) = (b)$ for some elements $a,b \in R$ iff $a = ub$ for some unit $u \in R$.
}

\begin{proof}
    ($\implies$) If $(a) = (b)$, $a \in (b)$ and $b \in (a)$. That is, $a = ub$ and $b = va$ for some $u,v \in R$.
    Thus, $ab = (ub)(va) = (uv)ab$, or $ab(1 - uv) = 0$.
    Hence, $uv = 1$, meaning that $u \in R$ is a unit.

    ($\impliedby$) Notice that $ca \in (a)$ can be written $c(ub) = (cu)b \in (b)$.
    Next, notice that $db = d(u^{-1}a) = (du^{-1})a \in (a)$.
\end{proof}

\prob{7.4.9}{
    Let $R$  be the ring of all continuous functions on $[0,1]$ and let $I$ be the collection of functions $f(x)$ in $R$ with $f(\frac{1}{3}) = f(\frac{1}{2}) = 0$.
    Prove that $I$ is an ideal of $R$ but it is not a prime ideal.
}

\begin{proof}
    We know that the ring of functions is commutative, so we need only show that $I$ is left ideal to show that it is ideal.
    This is easy to do since if $f \in I$ and $g \in R$, then
    \begin{align*}
        gf\left( \frac{1}{3} \right) = g\left(\frac{1}{3}\right)f\left(\frac{1}{3}\right) = g\left(\frac{1}{3}\right) \times 0 = 0 \\
        gf\left(\frac{1}{2}\right) = g\left(\frac{1}{2}\right)f\left(\frac{1}{2}\right) = g\left(\frac{1}{2}\right) \times 0 = 0 
    .\end{align*}
    Hence, $I$ is an ideal in $R$.

    It is easy to see that $I$ is not a prime ideal.
    Consider the linear functions $f(x) = x - \frac{1}{3}$ and $g(x) = x - \frac{1}{2}$.
    Clearly, $fg$ is identically zero at $\frac{1}{3}$ and $\frac{1}{2}$, but neither is zero at both.
    That, is $fg \in I$ but $f,g \not\in I$.
\end{proof}

\prob{7.4.17}{
    Let $x^3 - 2x + 1$ be an element of the polynomial ring $E = \mathbb{Z}[x]$ and use the bar notation to denote passage to the quotient ring $\mathbb{Z}[x]/(x^3 - 2x + 1)$.
    Let $p(x) = 2x^7 - 7x^5 + 4x^3 - 9x + 1$ and let $q(x) = \left( x - 1 \right)^{4}$.
}

a) Express each of the following elements of $\overline{E}$ in the form $\overline{f(x)}$ for some polynomial $f(x)$ of degree $\leq 2$: $\overline{p(x)},\,\overline{q(x)},\,\overline{p(x) + q(x)},\,\text{ and } \overline{p(x)q(x)}$.

$\rightarrow$ We define $\overline{p(x)}$ as the remainder when $p(x)$ is divided by $x^3 - 2x + 1$.
This gives us that 
\begin{align*}
    \overline{p(x)} &= -x^2 - 11x + 3 \\
    \overline{q(x)} &= 8x^2 - 13x + 5 \\
    \overline{p(x) + q(x)} &= \overline{\overline{p(x)} + \overline{q(x)}} = 7x^2 - 24x + 8 \\
    \overline{p(x)q(x)} &= \overline{\overline{p(x)}\,\overline{q(x)}} = 146x^2 - 236x + 90 
.\end{align*}

b) Prove that $\overline{E}$ is not an integral domain.

\begin{proof}
    We can write $x^3 - 2x + 1 = \left( x-1 \right)\left( x^2 + x - 1 \right) = \overline{0}$, which is the product of two irreducible polynomials in $\overline{E}$.
    Thus, we have found two zero divisors in $\overline{E}$, meaning that $\overline{E}$ is not an integral domain.
\end{proof}

c) Prove that $\overline{x}$ is a unit in $\overline{E}$.

\begin{proof}
    Notice that $x\left( -x^2 + 2 \right) = x^3 - 2x = -\left( x^3 - 2x + 1 \right) + 1 = \overline{1}$.
    Hence, $\overline{x}$ is a unit in $\overline{E}$.
\end{proof}

\prob{7.4.27}{
Let $R$ be a commutative ring with $1 \ne 0$.
Prove that if $a$ is a nilpotent element of $R$ then $1 - ab$ is a unit for all $b \in R$.
}

\begin{proof}
    Suppose that $a^{m} = 0$.
    Then, 
    \begin{align*}
        (1-ab)(1 + ab + \ldots + (ab)^{m-1}) &= (1 + ab + \ldots + (ab)^{m-1}) - ab(1 + ab + \ldots + (ab)^{m-1}) \\
        &= (1 + ab + \ldots + (ab)^{m-1}) - (ab + (ab)^2 + \ldots + (ab)^{m}) \\
        &= 1
    .\end{align*} 
    This is true regardless of our choice of $b$.
\end{proof}

\prob{7.5.1}{
    Fill in all the details in the proof of Theorem 15.
}

\textbf{Theorem:} Let $R$ be a commutative ring.
Let $D$ be any nonempty subset of $R$ that does not contain 0, does not contain any zero divisors and is closed under multiplication.
Then there is a commutative ring $Q$ with 1 such that $Q$ contains $R$ as a subring and every element of $D$ is a unit in $Q$.
The ring $Q$ has the following additional properties.

a) every element of $Q$ is of the form $rd^{-1}$ for some $r \in R$ and $d \in D$. 
In particular, if $D = R \backslash \{ 0 \} $ then $Q$ is a field.

\begin{proof}
   The ring of fractions is generated by the equivalence relation $r_1/d_1 ~ r_2/d_2$ if $r_1 d_2 = r_2 d_2$.
   This is equivalent to saying that $r_1 d_1^{-1} = r_2 d_2^{-1}$, so we could write the equivalence classes $r/d$ in an equivalent alternate representation as $r d^{-1}$.
   Note that each $d \in D$ must have an inverse by assumption.

   Furthermore, if $D = R \backslash \{ 0 \} $, then, since we have already proven that addition and multiplication in $Q$ are well defined, it remains to show that $Q$ follows the distribution laws and that $Q,\,Q^{\times}$ are abelian groups under addition and multiplication, respectively.
   Seeing that sets are groups under addition and multiplication is easy to see from the familiar definitions of the operations, and seeing that they follow the distributive laws is clear as well.
\end{proof}

b) (uniqueness of $Q$) The ring $Q$ is the ``smallest'' ring containing $R$ in which all elements of $D$ become units, in the following sense.
Let $S$ be any commutative ring with identity and let $\varphi: R \to S$ be any injective ring homomorphism such that $\varphi(d)$ is a unit in $S$ for every $d \in D$.
Then there is an injective homomorphism $\Phi: Q \to S$ such that $\Phi |_{R} = \phi$.
In other words, any ring containing an isomorphic copy of $R$ in which all the elements of $D$ become units must also contain an isomorphic copy of $Q$.

\begin{proof}
    
\end{proof}

\prob{7.5.2}{
Let $R$ be an integral domain and let $D$ be a nonempty subset of $R$ that is closed under multiplication.
Prove that the ring of fractions $D^{-1}R$ is isomorphic to a subring of the quotient field of $R$.
}

\begin{proof}
    Suppose that $F$ is the quotient field of $R$.
    Then we define $\varphi: D^{-1}R \to F$ such that $\varphi(d^{-1}r) = \frac{r}{d}$.
     
\end{proof}

\prob{7.5.4}{
Prove that any subfield of $\mathbb{R}$ must contain $\mathbb{Q}$.
}
 
\begin{proof}
    Any subfield of $\mathbb{R}$ must be contained in $\mathbb{R}$ and be a field itself under the same operations.
    Hence, it must contain the integers since $1 \in \mathbb{R}$, implying $(1) = \mathbb{Z} \subset \mathbb{R}$.
    We also must have the multiplicative inverse of each element.
    That is, if $n \in \mathbb{Z} \backslash \{ 0 \} $ is also in the subfield, then $\frac{1}{n} \in \mathbb{Q}$ is also in the subfield, and $\left( \frac{1}{n} \right)$ must be contained in the subfield.
    Since any rational number can be generated by some principal ideal $\left( \frac{1}{n} \right)$, implying that $\mathbb{Q}$ is contained in every subfield of $\mathbb{R}$.
\end{proof}

\end{document}
