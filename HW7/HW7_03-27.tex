\documentclass[12pt,a4paper]{article}

\input{../preamble_files/packages}
\input{../preamble_files/figures}
\input{../preamble_files/references}
\input{../preamble_files/shortcuts}

\pagestyle{fancy}
\lhead{Richard Whitehill}
\chead{MATH 513 -- HW 7}
\rhead{03/27/22}
\cfoot{\thepage~of~\pageref{LastPage}}

\newcommand{\prob}[2]{\textbf{#1)} #2}
\newenvironment{proof}
{
\textbf{\underline{Proof}} \\
}
{
\begin{flushright}
$\blacksquare$
\end{flushright}}

\setlength{\parskip}{\baselineskip}
\setlength{\parindent}{0pt}

\begin{document}

\prob{3.1.1}{Let $\varphi: G \rightarrow H$ be a homomorphism and let $E \leq H$. Prove that $\varphi^{-1}(E) \leq G$. If $E \trianglelefteq H$ prove that $\varphi^{-1}(E) \trianglelefteq G$. Deduce that $\ker{\varphi} \trianglelefteq G$}

\begin{proof}
Notice that since $E$ is a subgroup of $H$, then $\id_H \in E$, so it is clear that $\id_G \in \varphi^{-1}(E)$ since $\varphi(\id_G) = \id_H$. Now let $a,b \in \varphi^{-1}(E)$. Then, there exist $c,d \in E$ such that $\varphi(a) = c$ and $\varphi(b) = d$. Thus, since $cd \in E$, we have that $\varphi(a)\varphi(b) = \varphi(ab) = cd \in E$, or $ab \in \varphi^{-1}(E)$. Finally, we let $\varphi(a) = c \in E$. Then $\varphi(a)^{-1} = \varphi(a^{-1}) = c^{-1} \in E$ or $a \in \varphi^{-1}(E)$. This means that $\varphi^{-1}(E)$ is a subgroup of $G$ if $G$ and $H$ are homomorphic and $E$ is a subgroup of $H$. 

Furthermore, let us suppose that $E$ is a normal subgroup of $H$. Let $g \in G$ and $a \in \varphi^{-1}(E)$. Then consider $gag^{-1}$. We want to show that this is in $\varphi^{-1}(E)$. Notice that $\varphi(gag^{-1}) = \varphi(g)\varphi(a)\varphi(g^{-1})$. Obviously, $\varphi(g),\varphi(g^{-1}) \in H$ and $\varphi(a) \in E$. Thus, $\varphi(gag^{-1}) \in hEh^{-1} = E$ if we denote $h = \varphi(g)$. Hence, it immediately follows that $gag^{-1} \in \varphi^{-1}(E)$ and that $\varphi^{-1}(E) \trianglelefteq G$.
\end{proof}

$\rightarrow$ As a corollary: It is trivially true that $\{\id_H\} \trianglelefteq H$, so $\ker{\varphi} = \varphi^{-1}(\{\id_H\}) \trianglelefteq G$.

\prob{3.1.3}{Let $A$ be an abelian group and $B \leq A$. Prove that $A/B$ is abelian. Give an example of a non-abelian group $G$ containing a proper normal subgroup $N$ such that $G/N$ is abelian.}

\begin{proof}
Recall the definition of a quotient group: $A/B = \{aB | a \in A \}$. Suppose that $a,a' \in A$, then $aB \cdot a'B = aa'B$ and $a'B \cdot aB = a'aB$. Obviously, since $A$ is abelian, it follows that $a'aB = aa'B$, or that $A/B$ is abelian.
\end{proof}

Example: Consider the dihedral group for symmetries of a triangle $G = D_6$ where $N = \langle r \rangle$. Obviously $G$ is not abelian and $N$ is a normal subgroup of $G$. We then see that $G/N$ has two elements $s\langle r \rangle$ and $\langle r \rangle$, which is abelian.

\prob{3.1.4}{Prove that in the quotient group $G/N$, $(gN)^{\alpha} = g^{\alpha}N$ for all $\alpha \in \integers$.}

\begin{proof}
This is a fairly simple proof by induction. By definition, if $\alpha = 0$, then we get $N = N$, and if $\alpha = 1$, then $(gN)^{1} = gN$. Next, suppose that $(gN)^{\alpha} = g^{\alpha}N$ for some positive integer $\alpha$. Then $(gN)^{\alpha + 1} = gN \cdot (gN)^{\alpha} = gN \cdot g^{\alpha}N = gg^{\alpha}N = g^{\alpha+1}N$.

Reversing the proof is simple for negative integers $\alpha$. Notice that if we let $g = g_1^{-1}$ (since $g$ must have an inverse in the group $G$), then we get negative powers from the above for the element $g_1$.
\end{proof}

\prob{3.1.5}{Use the preceding exercise to prove that the order of the element $gN$ in $G/N$ is $n$, where $n$ is the smallest positive integer such that $g^n \in N$. Give an example to show that the order of $gN$ in $G/N$ may be strictly smaller than the order of $g$ in $G$.}

\begin{proof}
The order of $gN \in G/N$ is the smallest $n \in \integers^{+}$ such that $(gN)^{n} = g^nN = \id N = N$. At this point, it may be tempting to say that $g^n = \id$, and while this may hold true for some $n$, it does not necessarily produce the smallest possible value of $n$. We simply need $N$, which is a subgroup of $G$ such that $g^nN = N$. Since $N$ is a subgroup, it is closed under multiplication (i.e. for any $m \in N$ then $g^n m \in N$). Hence, we need $g^n \in N$, or alternatively, $n$, the order of $gN$, is the smallest positive integer such that $g^n \in N$.
\end{proof}

\prob{3.1.24}{Prove that if $N \trianglelefteq G$ and $H$ is any subgroup of $G$ then $N \cap H \trianglelefteq H$.}

\begin{proof}
We have previously proven that if $N,H \leq G$ that $N \cap H \leq H$. Thus, it remains to show that $h(N \cap H)h^{-1} \subseteq N \cap H$ for any $h \in H$. Let $n \in N \cap H$, then obviously, $n \in H$ and $hnh^{-1} \in H$. Hence, $N \cap H$ is a normal subgroup of $H$.
\end{proof}

\prob{3.1.36}{Prove that if $G/Z(G)$ is cyclic then $G$ is abelian.}

\begin{proof}
Recall $Z(G) = \{g \in G | \forall x \in G,~ gx = xg\}$ and $G/Z(G) = \{gZ(G)|g \in G\}$. Suppose that $G/Z(G)$ is cyclic. Then, if $z \in Z(G)$ any element of $G/Z(G)$ can be written as $x^az$. Now, let $g,h \in G$. Then $gz = x^az$ and $hz = x^bz$ for some integers $a,b$. Notice since $G/Z(G)$ is abelian, then $(x^az)(x^bz) = (x^bz)(x^az)$. Thus, $x^ax^bz = x^bx^az$ or $x^ax^b = gh = hg = x^bx^a$, implying that $G$ is abelian.
\end{proof}

\prob{3.2.1}{Which of the following are permissible orders for subgroups of a group of order 120: 1, 2, 5, 7, 9, 15, 60, 240? For each permissible order give the corresponding index.}

$\rightarrow$ If $H$ is a subgroup of $G$, then it must be true that $|H|\,\big|\,|G|$. The permissible orders of subgroups are then as follows: 1(trivially), 2, 5, 15, and 60. The indices for each corresponding order is defined to be $|G|/|H|$, and are given as follows, respectively: 120, 60, 24, 8, and 2.

\prob{3.2.4}{Show that if $|G| = pq$ for some primes $p$ and $q$ (not necessarily distinct) then either $G$ is abelian or $Z(G) = 1$.}

\begin{proof}
Recall that $Z(G)$ is a subgroup of $G$, so $|G| = |G/Z(G)||Z(G)|$. Thus, $|G/Z(G)| \in \{1,p,q,pq\}$. If $|G/Z(G)| = 1$, then $Z(G) = G$, implying that $G$ is abelian. Next, if $|G/Z(G)| \in \{p,q\}$ (a set of prime numbers), then it is a cyclic group, and furthermore, it follows that $G$ is an abelian group. Finally, suppose that $|G/Z(G)| = pq$, then $|Z(G)| = 1$, and since $\id \in Z(G)$ for all groups $G$, it follows that $Z(G)$ is the trivial subgroup.
\end{proof}

\prob{3.2.8}{Prove that if $H$ and $K$ are finite subgroups of $G$ whose orders are relatively prime then $H \cap K = 1$.}

\begin{proof}
Suppose that $H,K$ are finite subgroups of $G$ with $\gcd(|H|,|K|) = 1$. Recall that if $x \in G$, which is finite, then $|G| = n|\langle x \rangle|$. Then for each $h \in H$ and $k \in K$, we have $|H| = m_h|\langle h \rangle|$ and $|K| = l_k|\langle k \rangle|$ for some integers $m_h,l_k$. It is clear that since $H,K$ are subgroups of $G$ that they both contain the identity. If $h,k \not= \id$, then $\gcd(|\langle h \rangle|,|\langle k \rangle|) = 1$. Hence $h$ and $k$ are distinct. Thus, $H \cap K$ only contains the identity of $G$.
\end{proof}


\end{document}
